\section{Organisation}

La réunion de lancement du projet s’est tenue le mercredi 24 septembre à 11h en visioconférence sur Teams, en présence de Frédéric LANGEN, Jérôme VERGENDO, Olivier FESTOR, Maher BOUHEBBAL, Ewan DECIMA et Mathieu BREIT. \\

Une réunion hebdomadaire est prévue chaque jeudi à 14h entre les étudiants et les encadrants industriels. Ces rencontres permettront de suivre l’avancement du projet, de discuter des difficultés rencontrées et de répondre aux éventuelles questions des étudiants. Elles se tiendront en visioconférence, mais une rencontre en présentiel avec les encadrants industriels est également envisagée. \\

À l’issue de chaque réunion, un compte rendu rédigé par les étudiants sera transmis aux encadrants de la BNP ainsi qu’à l’encadrant académique. \\

Une réunion de mi-parcours aura lieu entre le 20 et le 27 novembre 2025, avec les mêmes participants que lors de la réunion de lancement. Son objectif sera d’évaluer conjointement l’avancement du projet, l’implication des différentes parties et la qualité de la communication entre acteurs. Afin de faciliter son déroulement, chaque participant devra préalablement compléter \textit{le formulaire PI bilan novembre 2025}. \\

Enfin, la soutenance finale est programmée pour le mardi 12 février 2026. Le rapport final devra être remis aux encadrants industriels et académiques le 3 février 2026. Une version préliminaire sera transmise mi-janvier aux encadrants de la BNP afin de recueillir leurs retours et d’intégrer d’éventuelles améliorations dans le rapport définitif.
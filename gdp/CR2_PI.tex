\documentclass[a4paper,11pt]{article}

% Encodage et langue
\usepackage[utf8]{inputenc}
\usepackage[T1]{fontenc}
\usepackage[french]{babel}

% Mise en page
\usepackage{geometry}
\geometry{margin=2.5cm}
\usepackage{lmodern}
\usepackage{setspace}
\onehalfspacing

% Couleurs et encadrés
\usepackage{xcolor}
\usepackage{tcolorbox}
\tcbuselibrary{listings,breakable}

% Titres
\usepackage{titlesec}
\titleformat{\section}{\large\bfseries}{\thesection.}{1em}{}
\titleformat{\subsection}{\normalsize\bfseries}{\thesubsection.}{1em}{}

% Tableaux simplifiés
\usepackage{array}

% Header du document
\title{\LARGE Compte rendu de réunion \\ \large BNP Paribas – 02/10/2025}
\date{}

\begin{document}

\maketitle

% Infos générales
\begin{tcolorbox}[colback=blue!5!white,colframe=blue!75!black,title=Informations générales]
\textbf{Date :} 02/10/2025 \\
\textbf{Lieu :} Visioconférence : Nancy/Luxembourg \\
\textbf{Objet :} Brainstorm sur les aspects techniques et organisationnels du projet industriel BNP Paribas-Luxembourg
\end{tcolorbox}

\vspace{0.5cm}

\section*{1. Contexte}
Cette réunion avait pour objectif de préciser les choix techniques liés à l’utilisation de modèles de langage, la stratégie de développement open source ainsi que l’organisation du projet à moyen terme.

\section*{2. Participants}
\begin{tcolorbox}[colback=gray!5!white,colframe=black!75!black,title=Participants]
\begin{tabular}{>{\bfseries}l l}
Côté BNP & Jérome VERGENDO, Frédéric LANGEN \\
Côté notre équipe & Maher BOUHEBBAL, Mathieu BREIT, Ewan DECIMA \\
Encadrant pédagogique & Olivier Festor
\end{tabular}
\end{tcolorbox}

\section*{3. Choix techniques}
Le projet reposera sur des solutions \textbf{open source gratuites}, notamment le LLM hébergé sur les serveurs de l'école. \\
L’entreprise n’a pas besoin d’héberger le modèle, car elle réutilisera le code et la documentation (RAG, prompts) en interne. 

\section*{4. Langues et interface}
Les articles de presse peuvent être rédigés dans plusieurs langues : \textbf{français, anglais, allemand, italien, espagnol}. L’anglais est à privilégier. Certaines langues comme le mandarin posent des difficultés de traduction. \\
L’\textbf{interface utilisateur sera en anglais}.

\section*{5. Gestion de projet}
Le \textbf{diagramme de Gantt} est actuellement une version prévisionnelle, basée sur le template proposé par l’entreprise. Il distinguera la partie prévue et la partie réelle.

\section*{6. Réunions à venir}
\begin{itemize}
    \item Pas de réunion la semaine prochaine en raison d’un séminaire.
    \item Prochaine réunion reportée à la semaine suivante.
    \item Une \textbf{réunion chez l’entreprise} est prévue courant novembre.
\end{itemize}

\section*{7. Gestion des accès}
L’entreprise n’a \textbf{pas besoin d’un accès direct au dépôt Git}. Le développement restera géré par notre équipe.

\section*{8. Prochaines étapes}
\begin{itemize}
    \item Stabilisation du Gantt.
    \item Préparation de la prochaine réunion.
    \item Préparation de la rencontre en novembre chez BNP.
\end{itemize}

\end{document}

\section{Contraintes}

\subsection{Budget}

Un budget pourra être mobilisé pour la réalisation du projet, sous réserve d'une justification technique claire et détaillée
de notre part. Cette proposition budgétaire devra ensuite être validée par les instances administratives de la BNP afin
de garantir sa conformité aux procédures internes et aux contraintes financières de l'entreprise.

\subsection{Limites imposées}

En raison du caractère confidentiel et sensible du secteur bancaire, l'accès aux données clients sera fortement limité.
Le projet devra donc être conçu avec un volume réduit de données réelles.
L'enjeu sera de faire en sorte que cela n'affecte pas la représentativité et la précision des cas d'usage.

De plus, pour des raisons organisationnelles et de sécurité, les outils internes de la BNP (CI/CD interne,
infrastructures web, ou autres environnements spécifiques) ne seront pas intégrés dans le projet. Cela implique de recourir
à des solutions alternatives et indépendantes.

\subsection{Risques}

L'utilisation de l'IA générative pour la classification et la justification de décision comporte plusieurs risques. Tout d'abord,
il existe un risque de mauvaise décision de classification (faux positifs ou faux négatifs), pouvant impacter
l'évaluation du profil client. L’IA est également exposée à des menaces de data poisoning, où des données biaisées ou
manipulées pourraient altérer la qualité des résultats. La pertinence et la cohérence des décisions constituent un autre
enjeu, certaines classifications pouvant manquer de justification claire ou de traçabilité. Enfin, le phénomène
d’hallucination de l’IA (production d’informations erronées ou inventées) représente un risque majeur pour la fiabilité
du système et son adoption par les équipes métiers.

